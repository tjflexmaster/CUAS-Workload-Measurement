\section{Conclusion and Future Work}
This paper has analyzed the role workload analysis plays in the optomization process of a semiautomated system. Results have shown that an effective modeling system can be based off of Moore finite state machines connected together via a DiTG. Preliminary analysis of the model revealed that communication is a primary cause of spikes in workload. This concept is benificial since data transfer can be executed effectively using automation allowing us to smooth out the peaks in our simulations.

For our next steps we have several basic plans. We will also be adding an actor to handle sense and avoidance operations. In addition to that it is necesarry to implement a way to evaluate the cost of errors in decision making. There will be certain tasks that cannot be automated because the cost of failure requires a human's undivided attention. First we will be undergoing a sensitivity study, followed by a field test. One we've verified that our system analyzes workload correctly We will branch our research into two separate directions. The first area will consist of generated an optimized GUI for the WiSAR system, hopefully to allow a single operator to take full control of the WiSAR system. While the second branch will be formulated a generalized model that will have application across the board for UAV systems.

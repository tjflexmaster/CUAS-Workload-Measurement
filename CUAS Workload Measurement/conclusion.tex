\section{Summary and Future Work}
This paper proposed a model-checking approach to analyzing human workload in an Unmanned Aerial System.  Humans and other autonomous actors were modeled as modified Mealy machines, yielding a directed graph representing team communication.  Workload categories were distilled from the literature, and the models of the actors and team were augmented so that specific workload metrics could be obtained using model-checking.  Preliminary analysis demonstrated a weak level of validity, namely, that the temporal workload profile was consistent with expected behavior for a set of well-understood situations.  For these scenarios, inter-actor communication was a primary cause of spikes in workload. 

For future work, we plan to  add an actor to represent a human performing the role of integrating the UAS into the National Air Space.  operations. A sensitivity study, followed by an experiment with real human users is needed to understand and justify the workload measures. One we have verified that our system analyzes workload correctly, it will be useful to design a GUI optimized to managing workload and to formulate a generalized model that will has application to other human-machine systems.

\section{Predicting Workload}
	In the interest of consolidating operators it is critical to find an accurate measurement that detects situations that exceed the capacity of a given human. One way to detect this is by building a map of each actor's workload as a function of time. JPF is excellent in this regard as it explores all possible paths the model can take and returns the ones that violate the model's criteria. By augmenting our model with the metrics described above we can identify all possible areas of high workload. Once these critical sections have been identified we will either be able to minimize them by increasing the autonomy of the system or by rearranging task protocols to balance sections of high workload with other lower workload areas.

There are three techniques that we need to undergo in the validation process of our workload measurements. The first is consistency. As we analyze the maps produced in JPF we will need to verify that deviations in the workload are pridictable results. For example there should be no spikes in workload when all the actors are at rest. Once we've verified that the workload measurements are consistent with our understanding we will execute a sensitivity study. During this study we will instigate random permutations into the model and verify that the workload mutations correlate with those permutations. This study will verify that our understanding of the workload metrics implemented is accurate. Finally we will need to initiate a human study to verify that our metrics do in fact correlate with workload in humans. In this study we will use situations that caused spikes, low points, and high points of workload. By running our subjects through these same situations and receiving their feedback we will be able to verify that our metrics do in fact measure workload.
\section{Workload Categories}
	In previous research four areas of workload have been well defined. They are Team, Algorithmic, Temporal and Cognitive.
\subsection{Team}
	Team Workload is the workload that results from the necesity of working with multiple agents in order to accomplish a task. It can be divided into two sections: group memory and communication. Group memory is 

\subsection{Algorithmic}
Algorithmic Workload results from the expected difficulty of bringing a task to completion. It is comprised of three phases: perceive, think, and act. During the perception phase the actor takes all active inputs and generates a list of his options. In the thinking phase the actor reviews the breadth of choices available and selects one. The workload in these two phases are directly proportional to the number of choices the actor has[ref?]. During the action phase the actor either follows through with the decision or disregards it. The workload in this section is entirely dependent on the length and difficulty of the chosen task[ref?]. This category of workload is highly dependent on the experience of the actor as an novice will have to examine each possible choice while an experienced agent will have gained a keen understanding of the best order to do things in[ref].

\subsection{Temporal}
Temporal workload deals with the stress of prioritized, infrequent and repetitious tasks. There are three main aspects that contribute to temporal workload[ref?]. First are the next state constraints. These consist of both timing deadlines as well as the ordering of when tasks are addressed. When a task is constrained by the time it needs to be completed by or the need to complete other tasks in order to respond to the given task it gives the individual an added sense of pressure during the completion. The second category is operational tempo. Operational tempo represents how frequently new tasks arrive. Low and High tempo both cause issues with workload because they frequently result in insufficient time to complete all the given tasks or tasks are forgotten[ref?]. This can also combine with the concept of fanout, or repetive tasks. 

\subsection{Cognitive}
Cognitive workload describes the difficulties associated with confusing neural signals. These issues come into play when dealing with multiple input channels such as an audio and video signal, in addition to memory accesses[ref?]. Cognitive workload can be divided into two categories: parallel sensing and sequential decision making. Parallel Sensing represents the difficulties associated with having multiple channels go high simultaneously. An example of this would be an individual hearing their name called while watching a silent movie. Both the auditory and visual channels are active but the individual is not overwhelmed because only the auditory channel required a response. Sequential decision making is the bottlenecking that occurs when multiple channels require a cognitive response. This results in spikes in the workload level because the channels can no longer run in parallel but must be handled individually[ref?].
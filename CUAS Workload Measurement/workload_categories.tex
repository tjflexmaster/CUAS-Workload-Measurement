\section{Workload Categories}

In this section, we present a brief review of the set of workload categories from which we distill a set of workload metrics.  We operationally accept Rasmussen's hierarchy of skill-based, rule-based, and knowledge-based activities~\cite{Rasmussen83}, and make the simplifying restriction of only considering rule-based and knowledge-based activities in our workload metrics.  This allows us to restrict attention to four general categories of workload metrics: cognitive, temporal, and algorithmic.  A fourth relevant workload category is the cost of maintaining team constructs like shared situation awareness~\cite{EliasFiore2011}, but we leave a discussion of team-related workload to future work.



\subsection{Cognitive}
Cognitive workload describes the difficulties associated with confusing managing various signals, decisions, and actions relevant to a particular task or goal~\cite{MorayEtAl91,lebiere2013cognitive,Goodrich2004,Chadwick2004}. We adopt Wickens' multiple resource theory where a human must manage multiple input parallel channels such as an audio and video signal, as well as manage limited ``storage" and accessibility in working memory while managing an inherently serial decision process~\cite{wickens2002multiple}. Given this model, we make the simplifying assumption that cognitive workload can be divided into two categories: parallel sensing and sequential decision making. Parallel sensing represents the difficulties associated with having multiple input channels ``go high'' simultaneously, meaning a human must use attentional resources to attend to complementary stimuli over different channels. An example of this would be an individual hearing their call sign on the radio while analyzing video. Both the auditory and visual channels are active but the individual is not overwhelmed because only the auditory channel required a response, but multiple signals may occur over the same channel at the same time causing conflict (i.e., two audio signals or two visual items needing attention). Sequential decision making occurs when a decision must be made, where we have adopted the assumption made by Wickens and supported by work in the psychology of attention~\cite{Pashler98} that ``bottle-neck" occurs when multiple channels require a cognitive response. 

\subsection{Algorithmic}
Algorithmic workload results from the expected difficulty of bringing a task to completion. Adopting a common model from artificial intelligence~\cite{Murphy00}, we assume that this is comprised of three phases: sense, plan, and act. During the sensing and perception phase, the actor takes all active inputs, interprets them, and generates a set of relevant decision-making parameters. In the planning phase the actor reviews the breadth of choices available and selects one, possibly using search or a more naturalistic decision-making process like recognition-primed decision-making~\cite{ZsambokKlein97} or a cognitive heuristic~\cite{GigerenzerTodd99}. We assume that the workload in these two phases is related to the number of choices the actor has, allowing us to use big-O analysis from computational complexity theory to describe the workload associated with sensing and planning. During the acting phase, the actor either follows through with the decision or disregards it. The workload in this section is entirely dependent on the length and difficulty of executing the plan. Before concluding, we note that workload is highly dependent on the experience of the actor~\cite{ZsambokKlein97}, but we leave a careful treatment of this to future work.


\subsection{Temporal}
Temporal workload deals with the scheduling of prioritized, infrequent, and repetitious tasks~\cite{DessoukyEtAl95,MorayEtAl91}. Various measures have been proposed, but we are most interested in those related to so-called fan-out, meaning the number of tasks that a single actor can manage \cite{Goodrich2010,OlsenWood2004,CrandallEtAl2005,Cummings2007}. There are two aspects of temporal workload that are directly related to the DiRG and DiTG formalisms presented earlier. The first aspect comprises the next state constraints. These consist of both timing deadlines as well as the ordering of when tasks are addressed. When a task is constrained by the time it needs to be completed by or the need to complete other tasks in order to respond to the given task, it limits the completion time or allowed interaction time of the actor, causing scheduling pressure~\cite{MauDolan2006}. The second category is operational tempo. Operational tempo represents how frequently new tasks arrive. Low and High tempo both cause issues with workload because they frequently result in insufficient time to complete all the given tasks or tasks are forgotten.  From a scheduling or queuing theory perspective, operational tempo impacts the arrival time or the neglect time.



\section{Workload Categories}

Workload is restricted to three general categories of metrics in this work: cognitive, temporal, and algorithmic.\footnote{A fourth relevant workload category is the cost of maintaining team constructs like shared situation awareness is future work \cite{EliasFiore2011}.}

Cognitive workload describes the difficulties associated with managing various signals, decisions, and actions relevant to a particular task or goal~\cite{MorayEtAl91,lebiere2013cognitive,Goodrich2004,Chadwick2004}. We adopt a simple form of Wickens' multiple resource theory~\cite{wickens2002multiple}, and make the simplifying assumption that cognitive workload can be divided into two categories: parallel sensing and sequential decision making. We further restrict the sensing channels to visual and auditory modalities, ignoring haptic.  Parallel sensing means that it is possible for a human to perceive complementary stimuli over different channels. An example of this would be an individual hearing their call sign on the radio while analyzing video. However, when multiple signals may occur over the same channel at the same time, this induces attentional workload for the human. Sequential decision making occurs when a decision must be made, where we have adopted the assumption made by Wickens and supported by work in the psychology of attention~\cite{Pashler98} that a ``bottle-neck" occurs when multiple channels either (a)~require a decision to be generated or (b)~exceed the limits of working memory. 

Algorithmic workload results from the difficulty of bringing a task to completion. Adopting a common model from artificial intelligence~\cite{Murphy00}, and consistent with Wickens' three stage multiple resource model, we assume that this is comprised of three phases: sense, plan, and act. During the sensing phase, the actor takes all active inputs, interprets them, and generates a set of relevant decision-making parameters. In the planning phase the actor reviews the breadth of choices available and selects one. The actor might use search or a more naturalistic decision-making process like recognition-primed decision-making~\cite{ZsambokKlein97} or a cognitive heuristic~\cite{GigerenzerTodd99}. In the acting phase the actor carries out the decision. Before concluding, we note that workload is highly dependent on the experience of the actor~\cite{ZsambokKlein97}, but we leave a careful treatment of this to future work.

\begin{comment}
We assume that the workload in these two phases is related to the number of choices the actor has, allowing us to use big-O analysis from computational complexity theory to describe the workload associated with sensing and planning; in this paper we make the unrealistic assumption that workload from computational complexity is $O(n)$, but include an explicit temporal component that allows us to detect when multiple decisions ``pile up" at the same time. During the acting phase, the actor either follows through with the decision or disregards it. The workload in this section is entirely dependent on the length and difficulty of executing the plan. 
\end{comment}

Temporal workload deals with the scheduling of prioritized, infrequent, and/or repetitious tasks~\cite{DessoukyEtAl95,MorayEtAl91}. Various measures have been proposed, but we are most interested in those related to so-called ``fan-out", meaning the number of tasks that a single actor can manage \cite{Goodrich2010,OlsenWood2004,CrandallEtAl2005,Cummings2007}. There are two particularly important aspects of temporal workload. First, when a task is constrained by (a)~the time by which the task must be completed or (b)~the need to complete other tasks before or after the given task then (c)~it causes scheduling pressure and workload~\cite{MauDolan2006}. The second aspect is operational tempo, which represents how frequently new tasks arrive.  From a scheduling or queuing theory perspective, operational tempo impacts workload by causing pressure to manage the rate of arrival and the response time of the decision.

\section{Workload Categories}

Workload categories.
4 Categories
	In previous research four areas of workload have been well defined. They are Team, Algorithmic, Temporal and Cognitive.
\subsection{Team}
	Team Workload is the workload that results from the necesity of working with multiple agents in order to accomplish a task. It can be divided into two sections: group memory and communication. Group memory is 

\subsection{Algorithmic}
Algorithmic Workload is the result of the difficulty of a task from start to completion. There are three phases these categories can be broken down into: perceive, think and act. During the perception phase the actor takes in all possible inputs and comes up with the list of options available given those inputs. The more options an actor has the greater the workload. Other factors that contribute to the workload of the decision making process are the difficulty of the intended course of action as well as the length of time that said course of action will take.

\subsection{Temporal}
Temporal workload deals with the stress of prioritized, infrequent and repetitious tasks. There are three main aspects that contribute to temporal workload. First are the next state constraints. These consist of both timing deadlines as well as the ordering of when tasks are addressed. When a task is constrained by the time it needs to be completed by or the need to complete other tasks in order to respond to the given task it gives the individual an added sense of pressure during the completion. The second category is operational tempo. Operational tempo represents how frequently new tasks arrive. Low and High tempo both cause issues with workload because they frequently result in insufficient time to complete all the given tasks or tasks are forgotten. This can also combine with the concept of fanout, or repetive tasks. Repetitive tasks contribute to the workload due

\subsection{Cognitive}
Cognitive workload describes the difficulties associated with confusing neural signals, this consists of both different channels as well as internal issues such as memory. Parallel Sensing represent the difficulties associated with having multiple channels go high simultaneously. An example of this would be someone's name being called while watching a silent movie. Both the auditory and visual channels are active but the individual is not overwhelmed. The other aspect to Cognitive workload is the necesity of sequential decision making. Conflicts arise between multiple channels each of those channels require decisions to be made based off the input. This results in spikes in the workload level.
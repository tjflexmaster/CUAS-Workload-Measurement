\section{Related Work}

This work is an extension of previous work which focused on modeling human
machine systems, specifically WiSAR.  This work extends this model to
incorporate the measurement of workload~\cite{gledhill2013modelinguas}.

Multiple resource theory plays a key role in how we are measuring workload~\cite{wickens2002multiple}. The multiple resource model defines four categorical dimensions that
account for variations in human task performance.  A task can be represented as a vector of these dimensions.  Tasks interfere when they share resource dimensions.  Using these vectors, Wickens defined a basic workload measure consisting of the task difficulty (0,1,2) and the number of shared dimensions.  Using this metric it is possible to predict task interference by looking at tasks which use the same resource dimensions.  Our model differs in that we do not explicitly define tasks, instead we use Actor state transitions which may imply any number of concurrent tasks.  The transition then informs us of which resources are being used and for how long.  

Threaded cognition theory states that humans can perform
multiple concurrent tasks that do not require executive processes~\cite{salvucci2008threaded}.  By making a
broad list of resource assumptions about humans, threaded cognition is able to
detect the resource conflicts of multiple concurrent tasks.  Our model differs from
threaded cognition theory in that it does not allow learning nor does our model distinguish
between perceptual and motor resources. In almost all other aspects our model behaves in a similar fashion.

Related work on temporal workload has attempted to predict the number of UAVs an operator can
control, otherwise known as {\em fan-out}~\cite{cummings2007predicting,OlsenWood2004,CrandallEtAl2005}.  This work used queuing theory
to model how a human responds in a time sensitive multi-task environment.  Queuing theory is helpful in determining the temporal effects of task performance by measuring the difference between when a task was received and when it was executed.  Actors can only perform a single transition at a time, similar to queuing theory, but it is possible for each state to take input from multiple concurrent tasks which differs from standard models of queuing theory.

ACT-R is a cognitive architecture which attempts to model human cognition and
has been successful in human-computer interaction applications~\cite{anderson2004integrated,lebiere2013cognitive}.  The
framework for this architecture consists of modules, buffers, and a pattern-matcher which in many ways are very similar to our own framework.  The major
difference is that ACT-R includes higher levels of modeling detail, such as memory access
time, task learning, and motor vs perceptual resource differences.  Our model exists at a higher level of abstraction.

Complementary work has been done using Brahms.
%Brahms is a powerful language that allows for more detail than our research required.
Brahms is a powerful language that allows for far more detail than we found our research required.
In addition at the time we started developing our model Brahms lacked some of the tools we needed for extracting workload data from the model. Because of this we found that Java in conjunction with JPF made a better match for our needs.

%% bare_conf.tex
%% V1.2
%% 2002/11/18
%% by Michael Shell
%% mshell@ece.gatech.edu
%%
%% NOTE: This text file uses MS Windows line feed conventions. When (human)
%% reading this file on other platforms, you may have to use a text
%% editor that can handle lines terminated by the MS Windows line feed
%% characters (0x0D 0x0A).
%%
%% This is a skeleton file demonstrating the use of IEEEtran.cls
%% (requires IEEEtran.cls version 1.6b or later) with an IEEE conference paper.
%%
%% Support sites:
%% http://www.ieee.org
%% and/or
%% http://www.ctan.org/tex-archive/macros/latex/contrib/supported/IEEEtran/
%%
%% This code is offered as-is - no warranty - user assumes all risk.
%% Free to use, distribute and modify.

% *** Authors should verify (and, if needed, correct) their LaTeX system  ***
% *** with the testflow diagnostic prior to trusting their LaTeX platform ***
% *** with production work. IEEE's font choices can trigger bugs that do  ***
% *** not appear when using other class files.                            ***
% Testflow can be obtained at:
% http://www.ctan.org/tex-archive/macros/latex/contrib/supported/IEEEtran/testflow


% Note that the a4paper option is mainly intended so that authors in
% countries using A4 can easily print to A4 and see how their papers will
% look in print. Authors are encouraged to use U.S. letter paper when
% submitting to IEEE. Use the testflow package mentioned above to verify
% correct handling of both paper sizes by the author's LaTeX system.
%
% Also note that the "draftcls" or "draftclsnofoot", not "draft", option
% should be used if it is desired that the figures are to be displayed in
% draft mode.
%
% This paper can be formatted using the peerreviewca
% (instead of conference) mode.
% If the IEEEtran.cls has not been installed into the LaTeX system files,
% manually specify the path to it:
% \documentclass[conference]{IEEEtran}
% To use A4 paper, add a4paper option as in
% \documentclass[conference,a4paper]{IEEEtran}


% setup page to suit conference specification using fancyhdr
\documentclass[conference,letterpaper]{IEEEtran}
\usepackage{fancyhdr}
\setlength{\paperwidth}{215.9mm}
\setlength{\hoffset}{-9.7mm}
\setlength{\oddsidemargin}{0mm}
\setlength{\textwidth}{184.3mm}
\setlength{\columnsep}{6.3mm}
\setlength{\marginparsep}{0mm}
\setlength{\marginparwidth}{0mm}

\setlength{\paperheight}{279.4mm}
\setlength{\voffset}{-7.4mm}
\setlength{\topmargin}{0mm}
\setlength{\headheight}{0mm}
\setlength{\headsep}{0mm}
\setlength{\topskip}{0mm}
\setlength{\textheight}{235.2mm}
\setlength{\footskip}{12.4mm}

\setlength{\parindent}{1pc}


% some very useful LaTeX packages include:

\usepackage{graphicx}  % Written by David Carlisle and Sebastian Rahtz
                        % Required if you want graphics, photos, etc.
                        % graphicx.sty is already installed on most LaTeX
                        % systems. The latest version and documentation can
                        % be obtained at:
                        % http://www.ctan.org/tex-archive/macros/latex/required/graphics/
                        % Another good source of documentation is "Using
                        % Imported Graphics in LaTeX2e" by Keith Reckdahl
                        % which can be found as esplatex.ps and epslatex.pdf
                        % at: http://www.ctan.org/tex-archive/info/
                        % NOTE: for dual use with latex and pdflatex, instead load graphicx like:
                        %\ifx\pdfoutput\undefined
                        %\usepackage{graphicx}
                        %\else
                        %\usepackage[pdftex]{graphicx}
                        %\fi
                        %
                        % However, be warned that pdflatex will require graphics to be in PDF
                        % (not EPS) format and will preclude the use of PostScript based LaTeX
                        % packages such as psfrag.sty and pstricks.sty. IEEE conferences typically
                        % allow PDF graphics (and hence pdfLaTeX). However, IEEE journals do not
                        % (yet) allow image formats other than EPS or TIFF. Therefore, authors of
                        % journal papers should use traditional LaTeX with EPS graphics.
                        %
                        % The path(s) to the graphics files can also be declared: e.g.,
                        % \graphicspath{{../eps/}{../ps/}}
                        % if the graphics files are not located in the same directory as the
                        % .tex file. This can be done in each branch of the conditional above
                        % (after graphicx is loaded) to handle the EPS and PDF cases separately.
                        % In this way, full path information will not have to be specified in
                        % each \includegraphics command.
                        %
                        % Note that, when switching from latex to pdflatex and vice-versa, the new
                        % compiler will have to be run twice to clear some warnings.

\usepackage{amsmath}   % From the American Mathematical Society
                        % A popular package that provides many helpful commands
                        % for dealing with mathematics. Note that the AMSmath
                        % package sets \interdisplaylinepenalty to 10000 thus
                        % preventing page breaks from occurring within multiline
                        % equations. Use:
                        %\interdisplaylinepenalty=2500
                        % after loading amsmath to restore such page breaks
                        % as IEEEtran.cls normally does. amsmath.sty is already
                        % installed on most LaTeX systems. The latest version
                        % and documentation can be obtained at:
                        % http://www.ctan.org/tex-archive/macros/latex/required/amslatex/math/

% Pdflatex produces superior hyperref results and is the recommended
% compiler for such use.

\usepackage{fancyhdr}
\usepackage{citesort}

\usepackage{comment}


% *** Do not adjust lengths that control margins, column widths, etc. ***
% *** Do not use packages that alter fonts (such as pslatex).         ***
% There should be no need to do such things with IEEEtran.cls V1.6 and later.


% correct bad hyphenation here
\hyphenation{op-tical net-works semi-conduc-tor IEEEtran}

\begin{document}
% paper title
\title{Modeling UASs for Role Fusion and Human Machine Interface Optimization}


% author names and affiliations
% use a multiple column layout for up to three different
% affiliations
\author{
\authorblockN{TJ Gledhill, Eric Mercer and Michael A. Goodrich}
\IEEEauthorblockA{Computer Science Department \\ Brigham Young University \\ Provo, UT}
}

\IEEEoverridecommandlockouts
\IEEEpubid{\makebox[\columnwidth]{978-1-4799-0652-9/13/\$31.00~\copyright2013
IEEE \hfill} \hspace{\columnsep}\makebox[\columnwidth]{ }}



% make the title area
\maketitle

% \thispagestyle{plain}


% insert page header and footer here for IEEE PDF Compliant
\fancypagestyle{plain}{
\fancyhf{}	% clear all header and footer fields
\fancyfoot[L]{}
\fancyfoot[C]{}
\fancyfoot[R]{}
\renewcommand{\headrulewidth}{0pt}
\renewcommand{\footrulewidth}{0pt}
}


\pagestyle{fancy}{
\fancyhf{}
\fancyfoot[R]{}}
\renewcommand{\headrulewidth}{0pt}
\renewcommand{\footrulewidth}{0pt}

\setlength{\topmargin}{1pt}
\setlength{\topskip}{1pt}
\setlength{\abovedisplayskip}{1pt}
\setlength{\belowdisplayskip}{1pt}



\begin{abstract}
Empty


%can provide new insight into human machine interactions through role.   UAS-enabled Wilderness Search and Rescue (WiSAR) %as a collection of roles running in parallel.   as well as an encoding of this model in Java. This yields a precise formalization of %the individual WiSAR roles and allows behavior of the roles to be model-checked by
%Java Pathfinder to establish bounds and trends in the models. Results from the modeling activity and model-checking in Java %Pathfinder indicate (a)~that it is necessary to clearly identify an appropriate level of modeling abstraction and (b)~that building %and evaluating the model provide insight into the best practices for real and UAS-enabled WiSAR teams.
\end{abstract}
\begin{IEEEkeywords}
model checking, human machine interfaces, Java PathFinder, Unmanned Aerial
Systems, Wilderness Search and Rescue, Workload measurement
\end{IEEEkeywords}


% For peer review papers, you can put extra information on the cover
% page as needed:
% \begin{center} \bfseries EDICS Category: 3-BBND \end{center}
%
% for peerreview papers, inserts a page break and creates the second title.
% Will be ignored for other modes.
\IEEEpeerreviewmaketitle

\section{Introduction}

Unmanned aerial systems (UASs), ranging from large military-style Predators to small civilian-use hovercraft, usually require more than one human to operate.  This is perhaps ironic, but when a UAS is part of a mission that requires more than moving from point A to point B, there are many different tasks that must be performed including operating the UAS, managing a payload (i.e., camera), managing mission objectives, etc.  Some even argue that this is desirable because different aspects of a mission are handled by humans trained for those aspects \cite{Murphy}, but since human resources are so expensive many argue that it is desirable to reduce the number of humans involved.

However, the question of {\em how} to reduce the number of humans while maintaining a high level of robustness is an open question.  Some progress has been made by including increasing levels of autonomy including automatic path-planning~\cite{}, automated target recognition~\cite{} etc., and careful human factors analyses have been performed to understand how these technologies impact workload, but with results that are often subtle and sometimes counterintuitive~\cite{}.

We argue that one reason for the limitations of prior work is that measures of workload is that the level of resolution for measuring workload is too low.  For example, although the NASA TLX dimensions include various contributing factors in workload (e.g., physical effort and mental effort), the temporal distribution of workload tends to be ``chunked" across a period of time.  Secondary task measures can provide a more detailed albeit indirect breakdown of available cognitive resources as a function of time, but without explanatory power for what in the task causes workload peaks and abatement.  Cognitive workload measures, including those that derive from Wickens' multiple resource theory, provide useful information about the causes of workload spikes, but these measures have not been widely adopted partly because of the difficulty in applying them to new systems.  Finally, measures derived from cognitive models such as ACT-R are providing more low-level descriptions of workload including a temporal history, but these approaches do not provide quantitative predictions for the resiliency of the system to deviations in the expected task evolution.

This paper presents a model of four human roles for a UAS-enabled wilderness search and rescue task, and is based on prior work on designing systems through field work and cognitive task analyses~\cite{Goodrich et al.}.  We identify seven {\em actors} in the team: the UAV, the operator and the operator's GUI, the video analyst and the analyst's GUI, the mission manager, and a role we call the parent search that serves to connect the UAS technical search team to the other components of the search enterprise.  In the next section, we present the formal model of each of these actors using finite state machines, and then discuss how the connections between these state machines defines what we call a {\em Directed Team Graph} (DiTG) that describes who communicates with whom and under what conditions.

We then identify a suite of possible workload measures based on a review of the literature, and augment the model to be able to encode specific metrics based on a subset of these measures.  Using the Java PathFinder model checker, we then create temporal profiles for each of the workload metrics and check consistency of the temporal profiles by associated workload peaks and abatements with likely causes. 


\section{Directed Team Graph (DiTG)}

In order to measure human workload within the context of our model \cite{gledhill2013modelinguas} we
have defined the following set of core components which allows us to correlate
the activity within our model to human workload.  We call this framework the
directed team graph (DiTG).

\subsection{Conceptual Model}
In previous work we represented WiSAR as a collection of directed role graphs
(DiRGs)\cite{gledhill2013modelinguas}.  See figure~\ref{fig:ditg}.  As is common
when modeling human-automation interaction we have decided to model the DiRGs usings Mealy
state machines~\cite{bolton2013litreview}.  The DiTG formalizes this collection
by defining the communication mediums which unite the said DiRGs.  By using
multiple resource theory \cite{wickens2002multiple} as a guide we can explicitly
define the channels over which this communication occurs.  See
figure~\ref{fig:ditg_detail}.  The Actor state transitions then use these
channels to broadcast and receive inter-Actor communication.  We hope to gain
insight into decreasing the system workload, and possibly combining roles, by
establishing metrics associated with the task model and model simulation.  These
metrics can then be used to determine if changes to the model represent a
decrease in operator workload.

\begin{figure}
\center
\setlength{\abovecaptionskip}{1mm}
\setlength{\belowcaptionskip}{1mm}
\setlength{\textfloatsep}{1mm}
\setlength{\floatsep}{1mm}
\includegraphics[height=2in]{ditg.png}
\caption{High Level DiTG}
\label{fig:ditg}
\end{figure}

\begin{figure}
\center
\setlength{\abovecaptionskip}{1mm}
\setlength{\belowcaptionskip}{1mm}
\setlength{\textfloatsep}{1mm}
\setlength{\floatsep}{1mm}
\includegraphics[height=1.9in]{ditg_detailed.png}
\caption{Detail view of DiTG: V is a Visual channel and A is an Audio channel}
\label{fig:ditg_detail}
\end{figure}

Formally, the models are the following mathematical structures:
 \begin{equation}
 	DiTG = (A, \Phi, \forall a_i \in A~ \exists \Phi_i \subset \Phi)
 \end{equation}

 \begin{equation}
 	Actor = (S, s_0, s_i, \Omega_A, \Sigma_A \subset \Phi, \Lambda_A \subset \Phi)
 \label{eq:actor}
 \end{equation}

 \begin{equation}
	State = (T_E, T_D)
 \label{eq:state}
\end{equation}

\begin{equation}
\begin{split}
	Transition = (\Omega_{input} \subset \Omega_A, \Sigma_{input} \subset \Sigma_A,\\
	\Omega_{input}_{values}, \Sigma_{input}_{values} \\
	\Omega_{output} \subset \Omega_A, \Lambda_{output} \subset \Lambda_A, \\
	\Omega_{output}_{values}, \Lambda_{output}_{values})
 \label{eq:transition}
 \end{split}
\end{equation}


where $S$ is a set of states, $s_0$ the start state,  $\Sigma_{A}$ the set of all Actor inputs, $\Sigma_{S}$ the set of all Simulator inputs, $\Lambda_{A}$ the set of all Actor outputs, and $T$ a transition matrix which specifies the outputs for any state transition.  $T$ may have multiple inputs and multiple outputs.

At this point our conceptual model is implementation agnostic.  Indeed, the abstraction allows us to group Actors, break Actors into sub-Actors, and use Actors to validate specific behaviors.  The model also allows for complex transitions between Actor states and the ability to enter normally unreachable state spaces using Events.  The model is also easily adapted to code which can be verified using model checking tools such as Java Pathfinder (JPF) which we will show in the following sections.


\subsection{Framework Components}
\subsubsection{Actors}
Actors represent the agents within the system, while an Actor may be any type of agent for the context of workload we assume that an Actor is human.  An Actor is made up of a current state, the set of all possible states, declarative memory, and channels and are represented within the model as finite state machines.

\subsubsection{States}
States contain a set of all possible transitions to other Actor states.  While states typically represent the performance of a set of tasks they can also represent such things as emotion, fatigue, and neglect which is expressed in the transitions.  In this way an Actor�s current state represents its current decision making paradigm.

\subsubsection{Transitions}
A transition is composed of a set of required declarative memory and channel values, a set of declarative memory and channel output values, an end state, and a duration.  A transition is considered enabled when all of its input requirements are met.  The duration represents the relative difficulty of the task(s) associated with the transition.  We rationalize this by assuming that all tasks are performed at a constant rate, thus more difficult tasks take longer.

\subsubsection{Declarative Memory }
This memory represents internal facts stored by an Actor and used in decision making.  This memory takes the form of internal variables within an Actor, transitions can look for specific values on these variables to determine if they are enabled.

\subsubsection{Channels} 
Channels represent physical communication mediums that exist between Actors.  Each channel has a type (audio or visual), a buffer, a source, and a target.  The channel type is associated with the modality dimension of multiple resource theory while the source and target represent the stages dimension.  The source being the response and the target being perception/cognition.  We assume that all channels have a constant bandwidth, the longer a channel is in use the more data being sent across the channel.

\subsubsection{How it works}
We represent a system as a directed team graph (DiTG).  This is a collection of Actors connected to one another by a set of channels.  Whenever the state of the system changes an Actor will petition, from its current state, a list of enabled transitions thus defining what decisions can be made.  The Actor may then activate one of these transitions.  Transitions have two main states, active and fired.  When chosen a transition is made active, after the specified duration the transition fires.  When a transition becomes active it creates temporary output values for declarative memory and channels.  These temporary values are then applied to the actual declarative memory and channels when the transition fires.  

\subsection{Actor vs Tasks}
This architecture focuses on the Actor itself and less on the tasks performed by the Actor.  The difference being that Actor states can account for conditions on the Actor which affect performance and increase workload which cannot be seen simply by taking a set of tasks and examining their combined resource requirements.  For our model we never explicitly define a single task.  Instead we define Actors, States, and Transitions.  Each transition defines its own perceptual, cognitive, response, and declarative resources (Salvucci) which are not restricted to performing a single task.  The state contains all the possible transitions an Actor can choose from given its current state representing the performance of multiple tasks and procedural resources.  The Actor acts as the final cognitive resource by choosing which transition to follow.  In this way we achieve multi-tasking not by defining specific tasks but by defining the choices an Actor can make and the resources affected by each choice.

\section{Workload Categories}

In this section, we present a brief review of the set of workload categories from which we distill a set of workload metrics.  We operationally accept Rasmussen's hierarchy of skill-based, rule-based, and knowledge-based activities~\cite{Rasmussen83}, and make the simplifying restriction of only considering rule-based and knowledge-based activities in our workload metrics.  This allows us to restrict attention to four general categories of workload metrics: cognitive, temporal, and algorithmic.  A fourth relevant workload category is the cost of maintaining team constructs like shared situation awareness~\cite{EliasFiore2011}, but we leave a discussion of team-related workload to future work.



\subsection{Cognitive}
Cognitive workload describes the difficulties associated with confusing managing various signals, decisions, and actions relevant to a particular task or goal~\cite{MorayEtAl91,lebiere2013cognitive,Goodrich2004,Chadwick2004}. We adopt Wickens' multiple resource theory where a human must manage multiple input parallel channels such as an audio and video signal, as well as manage limited ``storage" and accessibility in working memory while managing an inherently serial decision process~\cite{wickens2002multiple}. Given this model, we make the simplifying assumption that cognitive workload can be divided into two categories: parallel sensing and sequential decision making. Parallel sensing represents the difficulties associated with having multiple input channels ``go high'' simultaneously, meaning a human must use attentional resources to attend to complementary stimuli over different channels. An example of this would be an individual hearing their call sign on the radio while analyzing video. Both the auditory and visual channels are active but the individual is not overwhelmed because only the auditory channel required a response, but multiple signals may occur over the same channel at the same time causing conflict (i.e., two audio signals or two visual items needing attention). Sequential decision making occurs when a decision must be made, where we have adopted the assumption made by Wickens and supported by work in the psychology of attention~\cite{Pashler98} that ``bottle-neck" occurs when multiple channels require a cognitive response. 

\subsection{Algorithmic}
Algorithmic workload results from the expected difficulty of bringing a task to completion. Adopting a common model from artificial intelligence~\cite{Murphy00}, we assume that this is comprised of three phases: sense, plan, and act. During the sensing and perception phase, the actor takes all active inputs, interprets them, and generates a set of relevant decision-making parameters. In the planning phase the actor reviews the breadth of choices available and selects one, possibly using search or a more naturalistic decision-making process like recognition-primed decision-making~\cite{ZsambokKlein97} or a cognitive heuristic~\cite{GigerenzerTodd99}. We assume that the workload in these two phases is related to the number of choices the actor has, allowing us to use big-O analysis from computational complexity theory to describe the workload associated with sensing and planning. During the acting phase, the actor either follows through with the decision or disregards it. The workload in this section is entirely dependent on the length and difficulty of executing the plan. Before concluding, we note that workload is highly dependent on the experience of the actor~\cite{ZsambokKlein97}, but we leave a careful treatment of this to future work.


\subsection{Temporal}
Temporal workload deals with the scheduling of prioritized, infrequent, and repetitious tasks~\cite{DessoukyEtAl95,MorayEtAl91}. Various measures have been proposed, but we are most interested in those related to so-called fan-out, meaning the number of tasks that a single actor can manage \cite{Goodrich2010,OlsenWood2004,CrandallEtAl2005,Cummings2007}. There are two aspects of temporal workload that are directly related to the DiRG and DiTG formalisms presented earlier. The first aspect comprises the next state constraints. These consist of both timing deadlines as well as the ordering of when tasks are addressed. When a task is constrained by the time it needs to be completed by or the need to complete other tasks in order to respond to the given task, it limits the completion time or allowed interaction time of the actor, causing scheduling pressure~\cite{MauDolan2006}. The second category is operational tempo. Operational tempo represents how frequently new tasks arrive. Low and High tempo both cause issues with workload because they frequently result in insufficient time to complete all the given tasks or tasks are forgotten.  From a scheduling or queuing theory perspective, operational tempo impacts the arrival time or the neglect time.



\section{Metric Classes}
Conversion of the Workload theory into quantifiable measurements necessitated the creation of workload metric classes (see Figure~\ref{fig:WorkloadMetrics}). In the interest of finding areas to consolidate actors we left off measuring team workload. Given that cognitive workload describes the difficulties presented by managing resources such as memory and inputs we named that metric class resource. Algorithmic workload analyzes the difficulties presented by decisions giving rise to the decision metrics.


\begin{figure}[h]
\center
\setlength{\abovecaptionskip}{1mm}
\setlength{\belowcaptionskip}{1mm}
\setlength{\textfloatsep}{1mm}
\setlength{\floatsep}{1mm}
\includegraphics[height=2.5in]{WorkloadMetrics.png}
\caption{Workload in the model}
\label{fig:WorkloadMetrics}
\end{figure}

\subsection{Resource Metrics}
Cognitive workload is separated into both inter-actor communication and intra-actor memory analysis. We instrumented JPF to listen for recorded memory accesses to handle the intra-actor workload generated. JPF is then further instrumented to listen to all channel reads. By using these two listeners we gained the ability to maintain an accurate view of how the cognitive workload fluctuates over time.

\subsection{Temporal Metrics}
The DiTG lent itself to measuring temporal workload using three separate metrics. The op-tempo is measured by recording how many transitions occur over a course of the simulation. Tracking the rate at which inputs become active gives an accurate reflection for the arrival rate of data. The response time is calculated by measuring the time from when an input goes active to the point when it is read by the actor.

\subsection{Decision Metrics}
Algorithmic Workload can be broken down into the timing, the complexity of the algorithm, and the complexity of the solution. The timing is calculated by measuring the time when a transition is chosen till the outputs are posted. The point when the outputs go high represents the time when that task is complete and the actor has moved on to her next task. In this fashion we measure the time it takes to execute a given task. By counting the number of possible transitions we can track the number of choices the actor has and therefore measure the algorithmic complexity. The solution complexity is analyzed by counting the number of outputs a transition activates. This last measurement is not very precise since some tasks are more complex than others despite requiring the same number of actions. The level of abstraction does however give us sufficient accuracy to aid in our workload predictions while keeping our metrics at a manageable simplicity.
\section{Predicting Workload}
	In the interest of consolidating operators it is critical to find an accurate measurement that detects situations that exceed the capacity of a given human. One way to detect this is by building a map of each actor's workload as a function of time. JPF is excellent in this regard as it explores all possible paths the model can take and returns the ones that violate the model's criteria. By augmenting our model with the metrics described above we can identify all possible areas of high workload. Once these critical sections have been identified we will either be able to minimize them by increasing the autonomy of the system or by rearranging task protocols to balance sections of high workload with other lower workload areas.

There are three techniques that we need to undergo in the validation process of our workload measurements. The first is consistency. As we analyze the maps produced in JPF we will need to verify that deviations in the workload are predictable results. For example there should be no spikes in workload when all the actors are at rest. Once we've verified that the workload measurements are consistent with our understanding we will execute a sensitivity study. During this study we will instigate random permutations into the model and verify that the workload mutations correlate with those permutations. This study will verify that our understanding of the workload metrics implemented is accurate. Finally we will need to initiate a human study to verify that our metrics do in fact correlate with workload in humans. In this study we will use situations that caused spikes, low points, and high points of workload. By running our subjects through these same situations and receiving their feedback we will be able to verify that our metrics do in fact measure workload.
`\section{Results}
We are currently finding low workload, high workload, middle workload, delayed termination, and fast termination paths through the system using JPF. Using min and max duration flags it is possible to locate delayed and fast termination paths. Delayed paths indicate an inefficient team, while fast paths indicate an efficient team. While fast and slow paths are important to understand, low, high, and middle workload areas are, for the sake of this research, more important. Areas of low workload indicate an overabundance of human resources to accomplish a specific task. High workload is indicative of too few resources being allocated for a given task. Both aspects need to be minimized in order to optimize the current system. 

There are two different simulations that gave us interesting results. The first is when the Video Operator was able to identify the target during a flight without any complications occurring (see Figure~\ref{fig:WorkloadSim1}) For the first 40 time steps everything behaves as expected. The peaks result from periods of communication between actors as they exchange the information necessary to start the search. At time step forty we see a fascinating deviation from the norm. At that point the temporal workload dominates the system. This is a result of constant information passing between the GUIs and the operators. Since there is a constant passing of data between the machinery and the operators, it is only logical that the workload would increase substantially. However the results indicate that a weighting should be instigated to balance the three workload measures rather than allowing the single category to dominate the system. In our sensitivity study we will verify whether this is a fault in our metrics or if this one scenario lends itself to the distortion.

\begin{figure}[h]
\center
\setlength{\abovecaptionskip}{1mm}
\setlength{\belowcaptionskip}{1mm}
\setlength{\textfloatsep}{1mm}
\setlength{\floatsep}{1mm}
\includegraphics[height=2in]{WorkloadTargetSightingLabeled.png}
\caption{Workload over an uneventful flight}
\label{fig:WorkloadSim1}
\end{figure}

The second simulation we ran dealt with a situation where after a short period of flight time the battery rapidly fails. In this particular situation the operator was unable to respond quickly enough to land the UAV before it crashed(see Figure~\ref{fig:WorkloadSim2}). As with the previous simulation we saw an immediate spike in the temporal workload however in this simulation the workload decreased back to normal levels in just five time steps. The second spike that occurred indicated a sudden fluctuation in options among one of the actors which will have to be investigated further to verify if this is an accurate response or if a flaw in the model had slipped past the verification stage of development. Finally as would be expected when the UAV crashed there was a small spike in the workload before everything came to a halt. This last part of the simulation behaved precisely as expected.

\begin{figure}[h]
\center
\setlength{\abovecaptionskip}{1mm}
\setlength{\belowcaptionskip}{1mm}
\setlength{\textfloatsep}{1mm}
\setlength{\floatsep}{1mm}
\includegraphics[height=2in]{WorkloadCrashedLabeled.png}
\caption{Emergency battery failure simulation}
\label{fig:WorkloadSim2}
\end{figure}
\section{Related Work}

This work is an extension of previous work which focused on modeling human
machine systems, specifically WiSAR.  This work extends this model to
incorporate the measurement of workload~\cite{gledhill2013modelinguas}.

Multiple resource theory plays a key role in how we are measuring workload~\cite{wickens2002multiple}. The multiple resource model defines four categorical dimensions that
account for variations in human task performance.  A task can be represented as a vector of these dimensions.  Tasks interfere when they share resource dimensions.  Using these vectors, Wickens defined a basic workload measure consisting of the task difficulty (0,1,2) and the number of shared dimensions.  Using this metric it is possible to predict task interference by looking at tasks which use the same resource dimensions.  Our model differs in that we do not explicitly define tasks, instead we use Actor state transitions which may imply any number of concurrent tasks.  The transition then informs us of which resources are being used and for how long.  

Threaded cognition theory states that humans can perform
multiple concurrent tasks that do not require executive processes~\cite{salvucci2008threaded}.  By making a
broad list of resource assumptions about humans, threaded cognition is able to
detect the resource conflicts of multiple concurrent tasks.  Our model differs from
threaded cognition theory in that it does not allow learning nor does our model distinguish
between perceptual and motor resources. In almost all other aspects our model behaves in a similar fashion.

Related work on temporal workload has attempted to predict the number of UAVs an operator can
control, otherwise known as {\em fan-out}~\cite{cummings2007predicting,OlsenWood2004,CrandallEtAl2005}.  This work used queuing theory
to model how a human responds in a time sensitive multi-task environment.  Queuing theory is helpful in determining the temporal effects of task performance by measuring the difference between when a task was received and when it was executed.  Actors can only perform a single transition at a time, similar to queuing theory, but it is possible for each state to take input from multiple concurrent tasks which differs from standard models of queuing theory.

ACT-R is a cognitive architecture which attempts to model human cognition and
has been successful in human-computer interaction applications~\cite{anderson2004integrated,lebiere2013cognitive}.  The
framework for this architecture consists of modules, buffers, and a pattern-matcher which in many ways are very similar to our own framework.  The major
difference is that ACT-R includes higher levels of modeling detail, such as memory access
time, task learning, and motor vs perceptual resource differences.  Our model exists at a higher level of abstraction.

Complementary work has been done using Brahms.
%Brahms is a powerful language that allows for more detail than our research required.
Brahms is a powerful language that allows for far more detail than we found our research required.
In addition at the time we started developing our model Brahms lacked some of the tools we needed for extracting workload data from the model. Because of this we found that Java in conjunction with JPF made a better match for our needs.

Grant, Kraus, and Perlis have done exceptional work compiling formal approaches
to teamwork. 
They came up with the following categories: the persistence of joint intentions, the complexity of the roles of each agent, the individual actions required for teamwork, the joint commitment to a goal and cooperation, the collective intonations modeled as a multi-modal logical framework, and the subcontracting between cooperative agents.
%We want to quickly present is their list of categories followed by the category's emphasis: joint intentions, or rather the persistence of joint intentions; team plans, or rather the complexity of actions and the roles of agents; shared plans, or rather the individual intentions and actions needed for teamwork; cooperative problem solving, or rather the joint commitment to a goal and cooperation; collective intentions, or rather the formalization of collective intentions as multi-modal logical framework; and cooperative subcontracting, or rather the subcontracting between cooperative agents.
All of these approaches can be modeled within our simulator.
%It is important for the reader to see that all of these approaches could be modeled inside our simulator.
For example, joint intentions would be represented in actors containing complement transitions.
%Also, cooperative subcontracting would be represented in actors that contain transitions triggered by other actors.
While the formalisms are not explicit in our modeling, our informed modelers apply these approaches.

\section{Conclusion and Future Work}
This paper has analyzed the role workload analysis plays in the optimization process of a semi-automated system. Results have shown that an effective modeling system can be based off of Moore finite state machines connected together via a DiTG. Preliminary analysis of the model revealed that communication is a primary cause of spikes in workload. This concept is beneficial since data transfer can be executed effectively using automation allowing us to smooth out the peaks in our simulations.

Since optimal workload metrics are desired, we are proceeding with further
automation of metric analysis. While we may find a cumulatively middle path, it may be interspersed with workload peeks and valleys. These are like miniature instances of high and low workload, and they cause the same problems we are trying to avoid. JPF has been an exceptional tool in finding all the paths that the system can take. Since it produces nearly a million lines of data, finding paths with a stable workload must be automated.

For our next steps we have several basic plans. We plan to  add an actor to handle sense and avoidance operations. In addition to that it is necessary to implement a way to evaluate the cost of errors in decision making. There will be certain tasks that cannot be automated because the cost of failure is too high and such a critical role should be administered by a human. A sensitivity study, followed by a field test is needed to understand and justify the workload measures. One we have verified that our system analyzes workload correctly, we will branch our research into two separate directions. The first area will consist of designing an optimized GUI for the WiSAR system, hopefully to allow a single operator to take full control of the WiSAR.  The second branch will be to formulate a generalized model that will has application to other UASs.



% use section* for acknowledgement
% use section* for acknowledgement
\section*{Acknowledgment}
% optional entry into table of contents (if used)
% \addcontentsline{toc}{section}{Acknowledgment}
The authors would like to thank Neha Rungta of NASA Ames Intelligent
Systems Division for her help with JPF and Brahms. The authors would also like to thank the NSF IUCRC Center for Unmanned Aerial Systems, and the participating industries and labs, for funding the work. Further thanks go to Jared Moore and Robert Ivie for their help coding the model and editing this paper.


% can use a bibliography generated by BibTeX as a .bbl file
% standard IEEE bibliography style from:
% http://www.ctan.org/tex-archive/macros/latex/contrib/supported/IEEEtran/bibtex
\bibliographystyle{IEEEtran}
% argument is your BibTeX string definitions and bibliography database(s)
\bibliography{references}


% that's all folks
\end{document} 
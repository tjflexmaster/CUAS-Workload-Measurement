\section{Introduction}

Unmanned aerial systems (UASs), ranging from large military-style Predators to small civilian-use hovercraft, usually require more than one human to operate.  It is perhaps ironic that a so-called ``unmanned'' system requires multiple human operators, but when a UAS is part of a mission that requires more than moving from point A to point B, there are many different tasks that rely on human input including: operating the UAS, managing a payload (i.e., camera), managing mission objectives.  Some persuasively argue that this is desirable because different aspects of a mission are handled by humans trained for those aspects~\cite{MurphyBurke2010}. Human resources are expensive, and many others argue that it is desirable to reduce the number of humans involved.

However, the question of {\em how} to reduce the number of humans while maintaining a high level of robustness is an open question.  Some progress has been made by improving autonomy using, for example, automatic path-planning~\cite{WongBourgaultFurukawa2005,878915,pettersson2006probabilistic,QuigleyBarberEtAl2005,NelsonBarberMcLainBeard2006}, and automated target recognition~\cite{MorseEnghGoodrich2010,dasgupta2008multiagent,barber2006vision}.  However, careful human factors suggest that the impact of changes in autonomy are often subtle and difficult to predict, and this decreases confidence that the combined human-machine system will be robust across a wide range of mission parameters~\cite{KaberEndsley2004,chen2011supervisory,chen2007human}.

We argue that one reason for the limitations of prior work in measuring workload is that the level of resolution is too low.  For example, although the NASA TLX dimensions include various contributing factors to workload (e.g., physical effort and mental effort), the temporal distribution of workload tends to be ``chunked" across a period of time.  Secondary task measures can provide a more detailed albeit indirect breakdown of available cognitive resources as a function of time~\cite{kaber1999adaptive}, but with insufficient explanatory power for what in the task causes workload peaks and abatement.  Cognitive workload measures, including those that derive from Wickens' multiple resource theory~\cite{wickens2002multiple}, provide useful information about the causes of workload spikes, but these measures have not been widely adopted; one way to interpret this paper is as a step toward robust implementations of elements of these measures.  Finally, measures derived from cognitive models such as ACT-R are providing more low-level descriptions of workload which potentially include a temporal history~\cite{lebiere2013cognitive}, but these approaches may require a modeling effort that is too time-consuming to be practical for some systems.

This paper presents a model of four human roles for a UAS-enabled wilderness search and rescue (WiSAR) task, and is based on prior work on designing systems through field work and cognitive task analyses~\cite{Adams2008,GoodrichMorse2008}.  We first identify a suite of possible workload measures based on a review of the literature.  We then consider seven {\em actors} in the team: the UAV, the operator and the operator's GUI, the video analyst and the analyst's GUI, the mission manager, and a role we call the ``parent search'' which serves to connect the UAS technical search team to the other components of the search enterprise.  We present the formal model of each of these actors using finite state machines, and then discuss how the connections between these state machines defines what we call a {\em Directed Team Graph} (DiTG) that describes who communicates with whom and under what conditions.  We then augment the model to be able to encode specific metrics based on a subset of the measures identified in the literature.  Using the Java PathFinder model checker, we then create temporal profiles for each of the workload metrics and check consistency of the temporal profiles by associating workload peaks and abatements with likely causes. 


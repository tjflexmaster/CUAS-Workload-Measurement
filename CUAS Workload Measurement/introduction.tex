\section{Introduction}

Unmanned aerial systems (UASs), ranging from large military-style Predators to small civilian-use hovercraft, usually require more than one human to operate.  This is perhaps ironic, but when a UAS is part of a mission that requires more than moving from point A to point B, there are many different tasks that must be performed including operating the UAS, managing a payload (i.e., camera), managing mission objectives, etc.  Some even argue that this is desirable because different aspects of a mission are handled by humans trained for those aspects~\cite{MurphyBurke2010}, but since human resources are so expensive many argue that it is desirable to reduce the number of humans involved.

However, the question of {\em how} to reduce the number of humans while maintaining a high level of robustness is an open question.  Some progress has been made by including increasing levels of autonomy including automatic path-planning~\cite{WongBourgaultFurukawa2005,878915,pettersson2006probabilistic,QuigleyBarberEtAl2005,NelsonBarberMcLainBeard2006}, automated target recognition~\cite{MorseEnghGoodrich2010,dasgupta2008multiagent,barber2006vision} etc., and careful human factors analyses have been performed to understand how these technologies impact workload, but with results that are often subtle and difficult to predict, especially for a system designer~\cite{KaberEndsley2004,chen2011supervisory,chen2007human}.

We argue that one reason for the limitations of prior work in measuring workload is that the level of resolution is too low.  For example, although the NASA TLX dimensions include various contributing factors in workload (e.g., physical effort and mental effort), the temporal distribution of workload tends to be ``chunked" across a period of time.  Secondary task measures can provide a more detailed albeit indirect breakdown of available cognitive resources as a function of time~\cite{kaber1999adaptive}, but with insufficient explanatory power for what in the task causes workload peaks and abatement.  Cognitive workload measures, including those that derive from Wickens' multiple resource theory~\cite{wickens2002multiple}, provide useful information about the causes of workload spikes, but these measures have not been widely adopted partly because of the difficulty in applying them to new systems.  Finally, measures derived from cognitive models such as ACT-R are providing more low-level descriptions of workload which potentially include a temporal history~\cite{lebiere2013cognitive}, but these approaches do not provide quantitative predictions for the resiliency of the system to deviations in the expected task evolution.

This paper presents a model of four human roles for a UAS-enabled wilderness search and rescue (WiSAR) task, and is based on prior work on designing systems through field work and cognitive task analyses~\cite{Adams2008,GoodrichMorse2008}.  We identify seven {\em actors} in the team: the UAV, the operator and the operator's GUI, the video analyst and the analyst's GUI, the mission manager, and a role we call the parent search that serves to connect the UAS technical search team to the other components of the search enterprise.  In the next section, we present the formal model of each of these actors using finite state machines, and then discuss how the connections between these state machines defines what we call a {\em Directed Team Graph} (DiTG) that describes who communicates with whom and under what conditions.

We then identify a suite of possible workload measures based on a review of the literature, and augment the model to be able to encode specific metrics based on a subset of these measures.  Using the Java PathFinder model checker, we then create temporal profiles for each of the workload metrics and check consistency of the temporal profiles by associated workload peaks and abatements with likely causes. 

